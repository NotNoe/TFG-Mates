\chapter{Ecuación del calor}
\label{cap:heat}
\begin{resumen}
	En este capítulos desarrollaremos dos métodos numéricos, basado en las diferencias finitas, para aproximar la solución a la ecuación del calor en una  y dos dimensiones.
\end{resumen}

\section{Introducción}
En \emph{Théorie Analytique de la Chaleur}, \citeauthor{fourier} define la ley de la conducción térmica
\[
	\textbf{q}=-k\nabla T
\]

\section{Caso lineal}
En el caso unidimensional, mediante un cambio de variable podemos omitir la constante $k$ \citep[ver]{1dheat}, por lo que la ecuación podemos reducirla a

\begin{equation}\label{eq:1dheat}
	\frac{\partial u}{\partial t} = \frac{\partial ^2u}{\partial x^2}
\end{equation}

Pueden pedirse diferentes condiciones iniciales para asegurar la existencia y unicidad de \ref{eq:1dheat}, pero nosotros utilizaremos en concreto las condiciones
\begin{equation}
	u(x,0)=f(x) \hspace{20px} -\infty\leq x \leq \infty
\end{equation}

El problema de valor inicial que hemos definido tiene como dominio un rectángulo con uno de sus lados abierto hacia el infinito. Si fijamos $T>0$ tenemos ahora un rectángulo con el que trabajaremos. Definimos pues $B_T$ como la frontera del rectángulo y $D_T$ el interior de este, y por último definimos $D=D_T\cup B_T$.

\subsection{Existencia y unicidad}

Antes de estudiar la existencia y unicidad de la solución, necesitamos definir un concepto.
\begin{definicion}[Función continua a trozos]
	Una función es continua a trozos si es continua en todos sus puntos salvo en una cantidad finita de puntos.
\end{definicion}

Ahora enunciamos unas condiciones para la unicidad de las soluciones

\begin{teorema}[Unicidad]
	Sean $u$ y $v$ soluciones de la ecuación \ref{eq:1dheat} en $D_T$ continuas en $D$, si $u=v$ en $B_T$ entonces $u=v$ en$D$
\end{teorema}

\begin{teorema}[Unicidad extendida]
	Sean $u$ y $v$ soluciones de la ecuación \ref{eq:1dheat} en $D_T$ continuas a trozos en $D$ con una cantidad finita de discontinuidades acotadas, si $u=v$ en $B_T$ (excepto los puntos de discontinuidad) entonces $u=v$ en $D$
\end{teorema}

Estos teoremas nos dicen que bastan con comprobar que todas las soluciones coinciden en los límites del rectángulo para ver que la solución es única.

\begin{proof}
	Ver \citet{1dheat}, p.22
\end{proof}



\begin{teorema}[Existencia y unicidad]\label{teo:exis_uni_1dheat}
	Sean f, $\alpha$ y $\beta$ funciones continuas a trozos, la función
	
	\begin{multline}\label{eq:sol1dheat}
		u(x,t) = \int_{a}^{b}\theta(x-\xi,t)-\theta(x+\xi,t)f(\xi)d\xi \\
		- 2\int_{0}^{t}\frac{\partial \theta}{\partial x}(x, t-\tau)\alpha(\tau)d\tau+2\int_{0}^{t}\frac{\partial\theta}{\partial x}(x-1,t-\tau)\beta(\tau)d(\tau)
	\end{multline}
	
	donde $\theta(x,t)$ y $K(x,t)$ se definen como
	\[
		\theta(x,t)=\sum_{m=-\infty}^{\infty}K(x+2m,t) \hspace{15px} t>0
	\]\[
		K(x,t)=\frac{e^{\frac{-x^2}{4t}}}{\sqrt{4\pit}}\hspace{15px} t>0
	\]
	
	Es la única solución acotada del problema del valor inicial
	\begin{equation}\label{eq:pvi1dheat}
		\begin{cases}
			\frac{\partial u}{\partial t} = \frac{\partial ^2u}{\partial x^2} \hspace{20px}a<x<b, \hspace{10px} 0<t, \\
			u(x,0)=f(x), \hspace{15px} a<x<b, \\
			u(a,t)=\alpha(t), \hspace{15px} 0<t, \\
			u(b,t)=\beta(t), \hspace{15px} 0<t.
		\end{cases}
	\end{equation}
	
\end{teorema}

\begin{proof}
	Puede verse en \cite{1dheat} que en efecto \ref{eq:sol1dheat} es solución de la ecuación del calor, por lo que solo tenemos que preocuparnos por la unicidad. Esto es inmediato por el teorema \ref{teo:exis_uni_1dheat}, ya que las condiciones iniciales fijan el valor de cualquier solución en $B_T$.
\end{proof}

Ahora podemos asegurar que \ref{eq:pvi1dheat} tiene una única solución, por lo que podemos proceder a aproximarla con un método de diferencias finitas.

\subsection{Aproximación de la solución}



